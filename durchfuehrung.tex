\section{Durchführung}
\label{sec:Durchführung}
 
Während dieses Experimentes wird der Photostrom von Licht von vier verschiedenen Wellenlängen berechnet.
Die Photozelle wird so ausgerichtet, dass nur das Licht der gewünschten Wellenlängen in diese eintreten kann.
Dann wird die wird die Spannung des Beschleunigung- bzw. Abbremselements reguliert.
Sowohl beim roten, grünen und lila Licht wird die Bremsspannung von $0-2\, \unit{\volt}$ schrittweise erhöht.
In regelmäßigen Abständen wird in Abhängigkeit der Spannung der Photostrom abgelesen. 
Im Anschluss wird die Spannungsquelle umgepolt, sodass das Element als Beschleunigungsfeld dient.
Auch hier wird die Spannung von $0-2\,\unit{\volt}$ erhöht. Die Messwerte werden analog erfasst.
Für das orange Licht erfolgt die Messung ebenfalls analog, jedoch mit der Ausnahme, dass die Spannung jeweils zwischen $0-20\, \unit{\volt}$
angepasst wird.