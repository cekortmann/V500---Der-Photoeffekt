\section{Diskussion}
\label{sec:Diskussion}

Die in \autoref{sec:Auswertunga} bestimmte Grenzspannung $U_{\text{g}}$ verhält sich wider der Erwartungen. Es ist theoretisch anzunehmen, dass mit steigender Energie des Lichtes (Erhöhung der Frequenz), die Grenzspannung betragsmäßig 
zunimmt. Jedoch zeigt die Auswertung hier eine Abweichung, nach welcher $U_{\text{g,grün}}$ zwischen $U_{\text{g,rot}}$ und $U_{\text{g,lila}}$ liegen müsste. 
Diese Abweichung kann zum Teil dadurch erklärt werden, dass die Apparatur empfindlich gegenüber schon kleinen Erschütterungen war. So reichte es aus, auf dem Tisch zu schreiben, um die Messung zu beeinflussen.
In diesem fall liegt aber wahrscheinlich eher ein systematischer Fehler der Messung vor. So kann es zum Beispiel sein, dass die Skala des Picoamperemeters 
falsch abgelesen wurde.

Wenn das Verhältnis $\symup{\frac{h}{e_0}}$ mit dem theoretischen Wert $\symup{\frac{h}{e_0}} = 4.1356677 \cdot 10^{-15} \,\unit{\eV}$ verglichen wird, 
ergibt sich eine Abweichung von $60.97\,\%$. Der experimentelle Wert stimmt nicht im Rahmen der Messunsicherheit mit dem theoretischen Wert 
überein und ist, da er über eine Ausgleichsrechnung bestimmt wurde, recht ungenau, da nur drei Messungen 
in die Ausgleichsrechnung eingeflossen sind. Dies genügt statistisch nicht, um eine allgemein gültige Aussage zu treffen. 

Bei der Austrittsarbeit $A_{\symup{k}}$ kann kein Vergleich mit der Theorie hergestellt werden, da das Kathodenmaterial nicht bekannt ist. Der Wert der 
Austrittsarbeit liegt aber in der Größenordnung von bekannten Austrittsarbeiten von Materialien. 

